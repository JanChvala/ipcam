%=========================================================================
% 
\chapter{Conclusion}
\label{chap:chapter7}

The goal was to explore technologies for streaming from Android device and to create a system which would allow the user to easily stream the video content from the camera of a device to the web.

The first part of the thesis focuses on the description of WebRTC technology which is used in the resultant system for capturing the video and the streaming transmission. Then the Android operating system is described -- the target platform for the implementation. It is followed by the chapter about Google Cloud Messaging service, which is used for the streaming initiation. The chapter about the design and implementation details covers all the parts of the resultant system which are necessary for the understanding of the solution as a unit.

The created system was tested with various stream settings on different networks. It was proved that this system can be used in real conditions with good video quality and acceptable delays. Some WebRTC technological flaws were discovered during the testing and implementation phase, which indicates that the technology is still a draft and it is not yet completely finished. 

The resultant system was created for the demonstration of the purpose and possibilities of WebRTC technology. It is not a complex solution with a huge amount of settings, bulletproof security and stability. However, it can be used as a starting point for creating better Android IP camera application. There are a lot of features that could improve the current solution. It would be beneficial for the user to be able to change the streaming parameters remotely from the viewing page. Recording the stream would be also very useful. Finally, the authentication would have to be done better if it should go to production.

The thesis successfully demonstrated that the WebRTC technology can be used for creating the application for multimedia streaming although it has not been completely standardized yet.